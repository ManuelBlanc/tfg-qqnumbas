\documentclass{beamer}

\usepackage[T1]{fontenc}
\usepackage[utf8]{inputenc}
\usepackage[spanish]{babel}

\usepackage{amssymb,amsmath,mathtools}
\usepackage{array}

\usepackage{qqnumbas}

\beamertemplatenavigationsymbolsempty

\usetheme{Boadilla}

%% Contador que nos lleve el numero de pregunta. Automatizamos el titulo de los frames.
\newcounter{numpregunta}[section]
\addtobeamertemplate{frametitle}{%
	\def\insertframetitle{%
		\stepcounter{numpregunta}
		\ifx\insertsectionhead\empty
			\relax
		\else
			\insertsectionhead. Pregunta \thesection.\arabic{numpregunta}
		\fi
	}
}{}
\makeatletter
\CheckCommand*\beamer@checkframetitle{\@ifnextchar\bgroup\beamer@inlineframetitle{}}
\renewcommand*\beamer@checkframetitle{\global\let\beamer@frametitle\relax\@ifnextchar\bgroup\beamer@inlineframetitle{}}
\makeatother


%% Metadatos
\title[PR-Prueba]{Preguntas Rápidas de prueba para importar a Numbas}
\author[M. Blanc]{Manuel Blanc}
\institute[UAM]{Universidad Autónoma de Madrid}

\begin{document}
\maketitle

\section{Pruebas}

\begin{frame}
\quickquestion{%
name/.expanded = {PR.\jobname.\thesection.\thenumpregunta},
prompt = {%
	Esta pregunta es muy basica. Tiene un poco de matematicas: $2\times 3$ y $B$.
	Pero no puede tener varios parrafos!!!
},
choices={
	{Eleccion con matematicas: $e^{i\pi} + 1 = 0$},
	{Eleccion con puro texto},
	{Eleccion una matriz:
	$$\left\{\begin{array}{rrrr}
		42x	& 42y	& z 	& = 0	\\
		37x	&    	&   	& = 0	\\
		   	& 2y 	& +z	& = 0	\\
		2x 	&    	& z 	& = 0	%%
	\end{array}\right\}
	$$},
	{Nada aqui de momento}
},
answer={c},
}
\end{frame}


\begin{frame}
\quickquestion{%
name/.expanded = {PR.\jobname.\thesection.\thenumpregunta},
prompt = {%
	Esta pregunta es muy basica. Tiene un poco de matematicas: $2\times 3$ y $B$.
	Pero no puede tener varios parrafos!!!
},
choices={
	{Eleccion con matematicas: $e^{i\pi} + 1 = 0$},
	{Eleccion con puro texto},
	{Eleccion una matriz:
	$$\left\{\begin{array}{rrrr}
		42x	& 42y	& z 	& = 0	\\
		37x	&    	&   	& = 0	\\
		   	& 2y 	& +z	& = 0	\\
		2x 	&    	& z 	& = 0	%%
	\end{array}\right\}
	$$},
	{Nada aqui de momento}
},
answer={c},
}
\end{frame}


\end{document}

